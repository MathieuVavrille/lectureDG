\documentclass[a4paper, 11pt]{article}

\usepackage[utf8]{inputenc}
\usepackage{amsmath,amssymb}
\usepackage{epsfig}  
\usepackage{setspace}

\voffset -0cm
\hoffset 0.0cm
\textheight 22cm
\textwidth 16cm
\topmargin 0.0cm
\oddsidemargin 0.0cm
\evensidemargin 0.0cm


\title{\bf{TP2 \\ Histogram manipulation}}
\author{}
\date{}


%%%%%%%%%%%%%%%%%%%%%%%%%%%
%%% Debut du document %%%%%
%%%%%%%%%%%%%%%%%%%%%%%%%%%
\begin{document}
\maketitle

\section*{\bf Histogram analysis and transformation}

\subsection*{\bf Exercise 1. \rm Histogram analysis and export}

\par We want to improve the images \texttt{objects-dark.pgm} and \texttt{len\_dark.pgm}.

\begin{enumerate}
	\item Go back to the code from TP1 to load a PGM image.
	\item Build a small program that loads an image and that builds its normalized histogram. You will export this histogram into a text file containing a table whose first column is the intensity, and whose second column is [number of pixels which have this intensity] / [Total number of pixels].
	\item Use \texttt{gnuplot} to generate the drawing of this histogram (cf Appendix). What do you observe in the histograms of the considered PGM pictures?
\end{enumerate}

% -----------------------------------
\subsection*{\bf Exercise 2. \rm Transformation}

In the following, we suppose that the histogram intensities are mapped from $[0,M]$  to $[0,1]$.

\par Implement the following histogram transformations: 
	\begin{itemize}
		\item Histogram inversion
		\item Gamma-correction :  $i' = i^\frac{1}{\gamma}$. Check its behavior for $\gamma=2.2$ and $\gamma=\frac{1}{2.2}$ \footnote{Hint: $a^i$ = \texttt{pow(a,i)}, don't forget the \texttt{\#include<math.h>}}
		\item Linear interpolation $[a,b]\subset[0,1]\rightarrow [0,1]$
	\end{itemize}
Every times, you will illustrate the transformation by:
\begin{itemize}
	\item A \texttt{gnuplot} drawing of the histogram transformation function $\phi$.
	\item Two drawing (before/after) of the histograms.
	\item The modified PGM picture.
\end{itemize}


% -----------------------------------
\subsection*{\bf Exercise 3. \rm Equalization}

\par As presented during the lesson, our aim is to build a transfer function $\phi$ between the picture histogram (seen as an empirical probabilistic distribution $P(i)$), toward a targeted distribution. Here, we want to maximize the entropy of the image by targeting a uniform distribution.

\par Given an histogram over the intensities of the interval $[0,M]$, we want a distribution such that the probability of each gray level $i'=\phi(i)$ is $P'(i')=\frac{1}{M}$.

\par In the case of continuous variables and distributions and for an increasing transformation $\phi$, we have:
\begin{equation}
  P'(\phi(i)) = P(i)\frac{di}{di'}
\end{equation}
Thus,
\begin{equation}
  di' = M.P(i).di
\end{equation}
And,
\begin{equation}
  \phi(i) =  M \int_0^i P(\omega)d\omega
\end{equation}

\par In the discrete case, the transformation $\phi$ is the following:
\begin{equation}
  \phi(i) = M\frac{\sum_{j=0}^{i} hist(j)}{\sum_{j=0}^M hist(j)}
\end{equation}

{\bf Implement the histogram equalization transformation previously described and test this transformation over several pictures (with the corresponding \texttt{gnuplot} drawings of $\phi$ and their histograms before/after transformation)}.


% -------------------------------------------------------------------------
% -------------------------------------------------------------------------
% -------------------------------------------------------------------------
\newpage
\section*{\bf Image segmentation by using the histogram}

\subsection*{\bf Exercise 1. \rm Naive Threshold (fr: Seuillage naif)}

\begin{itemize}
	\item Build a small program that loads a picture, computes its histogram and threshold the initial image according to a given intensity.
\end{itemize}



\subsection*{\bf Exercise 2. \rm Variance minimization}

\par We will focus on different techniques to automatize the search of a \emph{good} threshold.

\par Fisher's method aims at minimizing the intra-class variance of the two classes (objects / background). More formally, let us consider a normalized histogram $H$ whose intensities are in $[0,M]$. Let us consider a threshold $t$ defining 2 classes (\emph{B=background, O=object}) of values in the histogram. We want to minimize the functional:
\begin{equation}
  \sigma^2_{intra}(t) = n_B(t)\sigma^2_B(t) + n_O(t)\sigma^2_O(t)
\end{equation}
where
\begin{align}
  n_B(t)  &= \sum_{i=0}^{t-1} H(i)\\
  n_O(t)  &= \sum_{i=t}^{M} H(i)\\
\mu_B(t) &= \frac{1}{n_B(t)} \sum_{i=0}^{t-1} H(i).i\\
\mu_O(t) &= \frac{1}{n_O(t)} \sum_{i=t}^{M} H(i).i\\
\sigma^2_B(t) &= \frac{1}{n_B(t)} \sum_{i=0}^{t-1}(i-\mu_B(t))^2 H(i)\\
\sigma^2_O(t) &= \frac{1}{n_O(t)} \sum_{i=t}^{M}(i-\mu_O(t))^2 H(i)
\end{align}


{\bf Question 1} Write a function which computes $\sigma^2_{intra}(t)$. Then, for a given image, write down a program which computes $argmin_{t\in{0..M}}(\sigma^2_{intra}(t))$, and use the corresponding threshold.

\smallskip
\par Otsu observed that minimizing the intra-class variance is equivalent to maximizing the variance between the classes defined by:
\begin{equation}
   \sigma^2_{entre}(t) = n_B(t)n_O(t)\left(\mu_B(t) - \mu_O(t) \right ) ^2
\end{equation}

{\bf Question 2} Similarly, write down a program that maximizes $\sigma^2_{entre}(t)$. You should not find another optimal threshold value, but you should compute it more effectively. To be effective, do not forget to update incrementally  $n_i$ et $\mu_i$, for example by using the following identities:
\begin{align}
  n_B(t+1) & = n_B(t) + H(t+1)\\
  \mu_B(t+1)& = ....
\end{align}


\newpage
\section*{\bf Appendix - \rm Crash-course \texttt{Gnuplot}}

Given a text file with tabular data (several columns, separated by spaces or tabs..) \texttt{toto.txt}
\begin{itemize}
\item Display a graph where abscissa/ordinate  are mapped to first/third column  
\begin{verbatim}
  plot "toto.txt" using 1:3 with points
\end{verbatim}

\item Same graph with lines and points between datum
\begin{verbatim}
  plot "toto.txt" using 1:3 with linespoints
\end{verbatim}

\item Y-axis is the sum of the second and third column and we limit the x-axis range to $[0,10]$
\begin{verbatim}
  plot [0:10] "toto.txt" using 1:($3+$2) with linespoints
\end{verbatim}

\item Labels
\begin{verbatim}
  set xlabel "Abscisse"
  set ylabel "Ordonnee"
\end{verbatim}

\item The next ``plot'' command will output the graph in a PDF file (with enhanced and color properties)
\begin{verbatim}
  set terminal pdf enhanced color
  set output "glop.pdf"
\end{verbatim}

\item To use the X11 terminal (default one)
\begin{verbatim}
  set terminal X11
  unset output
\end{verbatim}

\item Need some help on the plot command
\begin{verbatim}
  help plot
\end{verbatim}

\end{itemize}

\end{document}

