\documentclass[a4paper, 11pt]{article}

\usepackage{hyperref}
\voffset -0cm
\hoffset 0.0cm
\textheight 23cm
\textwidth 16cm
\topmargin 0.0cm
\oddsidemargin 0.0cm
\evensidemargin 0.0cm

\usepackage{epsfig}
\usepackage{setspace}
\usepackage{fancyheadings}
\usepackage{amsmath}
\usepackage{amssymb}
\usepackage{graphicx}
\usepackage{url}

\title{}
\author{}
\date{}

\begin{document}

\begin{center}
	\LARGE \textbf{TD12: Height maps and Mesh visualization}
\end{center}

\bigskip
\par In this TP, the objective is to practice a bit with triangular
mesh data-structures and 3D visualization using \texttt{DGtal}
viewer. The final objective is to render height maps as 3D meshes with
colorimetric information.

\section{Preliminaries}

Visualization will be performed by \texttt{DGtal}. If you use the
compiled library (``dcoeurjo'' account), the viewer is enabled by
default. If you use you own \texttt{DGtal} install, make sure that you
have compiled the library with \texttt{WITH\_QGLVIEWER} flag enables
(e.g. \texttt{cmake .. -DWITH\_QGLVIEWER=true}). You would need to
have Qt and QGLViewer installed in your system.

Please also checkout the last release of the \texttt{DGtalSkel} folder. The
file \texttt{image2mesh.cpp} gives examples of the \texttt{Viewer3D}
usage.

First, compile this example and when executing it, you'd see a OpenGL
window with three triangles and two cubes.

For this TP, you just need to know how to display triangles:
\begin{verbatim}
      Z3i::RealPoint p1(1.0,0.0,0.0),
        p2(0.0,1.0,0.0),
        p3(0.0,0.0,1.0);
      viewer.addTriangle(p1,p2,p3);
      viewer  << Viewer3D<>::updateDisplay;
\end{verbatim}


\section{Mesh data-structure and visualization}



\section{Normal map rendering and curvature estimation}





\end{document}
