\documentclass[a4paper, 11pt]{article}

\usepackage{hyperref}
\voffset -0cm
\hoffset 0.0cm
\textheight 23cm
\textwidth 16cm
\topmargin 0.0cm
\oddsidemargin 0.0cm
\evensidemargin 0.0cm

\usepackage{epsfig}  
\usepackage{setspace}
\usepackage{fancyheadings}
\usepackage{amsmath}
\usepackage{amssymb}
\usepackage{graphicx}
\usepackage{url}

\title{}
\author{}
\date{}

\begin{document}

\begin{center}
	\LARGE \textbf{TD11: Convex Hulls and Digital Convex Hulls}
\end{center}

\bigskip
\par In this TP, the idea is to implement a convex hull algorithm and to experiment its complexity (number of edges) on digital objects.

% -------------------------------------------------------
\section*{Exercise 1 \rm Orientation predicate}

{\bf Question:} Implement the $Orientation(p,q,r)$ predicate as discussed in the lecture.
	\begin{itemize}
	\item To avoid numerical issues, we encourage you to consider points in $\mathbb{Z}^2$ on a digital domain (instead of $\mathbb{R}^2$). Furthermore, it would help outputs since you will be able to reuse \textsc{DGtal} board mechanism. 
	\item To display a line in \texttt{DGtal} between \texttt{Points} $p$ and $q$, you can use the following method of the \texttt{Board} class: \texttt{board.drawLine( p[0], p[1], q[0], q[1]);}
	\end{itemize}

{\bf Question:} Test the $Orientation$ predicate with an implementation of the segment-segment intersection detection (cf lecture). Experiment your intersection test on all cases (regular intersection, alignement, no intersection, intersection point is a vertex, ...).


\bigskip
\bigskip
\bigskip

\par The rest of the TP focuses on the implementation and the experimentation of a \emph{convex hull algorithm}. You can choose any variant of the algorithm you want to implement. At the end we would like you:
	\begin{itemize}
	\item To test the convex hull construction on point sets defined by the digitization (at a given resolution $h$) of a disc (cf TP5) or an ellipse (cf DM) defined as digital sets.
	\item To plot (using \texttt{gnuplot}) the number of edges when $h\rightarrow 0$ in log-scale. The aim is to observe the $N^{2/3}$ behavior we discussed in the lecture for convex hull in $N\times N$ domains (also, express the relationships between $N, h$ and the slope of the best fitting straight line in log-scale) ?
	\end{itemize}


\par The first variant is a Graham's scan implementation on the point set ($O(n.logn)$), the second variant is based on the Melkman's algorithm on the contour points (extracted using a contour tracker) in $O(n)$. Both should only use the $Orientation(p,q,r)$ predicate and point coordinate comparisons.


\section*{Exercise 2 (Rookie Mode) - \rm Convex hull}

\noindent {\bf Questions:}
\begin{itemize}
	\item Implement the Graham's scan algorithm as described in the lecture:
		\begin{itemize}
		\item First, sort the points by polar angle (cf the \texttt{qsort} C function man page  or the C++ \texttt{std::sort} to do the sort). As discussed in the lecture, you would just have to replace the comparison function/functor by the Orientation predicate with fixed $p_0$.
		\item The second step consists in a simple stack based removal (using for example \texttt{std::queue}, cf previous TP).
		\end{itemize}
\end{itemize}

\par Note that Graham's scan can be speed up just considering border point and not interior points (for disc/ellipse experiment).


\section*{Exercise 2 (Advanced Mode) - \rm Convex hull}

\noindent {\bf Questions:}
\begin{itemize}
	\item To implement the Melkman's algorithm, you will first have to construct a sequence of points but you should have such function from previous TP. Implement Melkman's technique (cf lecture) using the \texttt{std::deque} as data structure to store the convex hull vertices. 
\end{itemize}

\par The overall code is not more complex than Graham's one but you'll have to dig previous TP for finding the appropriate border extraction code from a digital set.

\end{document}

